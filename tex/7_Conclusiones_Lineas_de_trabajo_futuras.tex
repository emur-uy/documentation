\capitulo{7}{Conclusiones y Líneas de trabajo futuras}


\subsection{Conclusiones}
La confección del presente TFG me ha permitido robustecer algunos conocimientos previos, así como también acercarme a técnicas y herramientas de las que manejaba vagas ideas y que me fueron seduciendo conforme el desarrollo del proyecto avanzaba. Asimismo, poner mi bagaje académico al servicio de una comunidad me interpeló profundamente confirmándome que cada persona puede configurarse como un agente de cambios positivos para la sociedad toda.

A continuación, detallo algunas de las conclusiones a las que he arribado a partir de este significativo proyecto en mi trayectoria académica-profesional.

\begin{itemize}
\item A través de la combinación de múltiples entornos de desarrollo, como Golang y Python, y diversas plataformas, se logró crear el backend y el chatbot para una futura aplicación que integra una serie de funciones que pretenden colaborar a la mejora de la calidad de vida de las personas afectadas por la EM.

\item Por tratarse del backend de una aplicación que será destinada a una organización sin fines de lucro, se puso especial énfasis en el ámbito económico procurando minimizar los costes de desarrollo. Ello implicó el uso de software y plataformas en su mayoría gratuitos sin afectar la funcionalidad o sacrificar bondades, lo que reafirma que es posible crear soluciones costo-efectivas.

\item Por otro lado, la tolerancia a alta concurrencia sin provocar fallos, así como el exiguo espacio de memoria requerido para el funcionamiento del backend, dieron como resultado una solución robusta que ofrece una buena experiencia de usuario.

\item Amerita explicitar que el desarrollo de la aplicación móvil como tal insumirá tiempo y recursos, desafiándome a concretarla para poder cumplir con mi propósito de donarla a EMUR.

\item Por último pero no menos importante, en lo referente a los objetivos personales, el desarrollo del proyecto supuso un proceso de aprendizaje continuo en el que pude adquirir nuevas habilidades tanto en el ámbito de la programación y desarrollo de software, como en el metodológico concerniente a la elaboración de esta memoria y a la documentación formal requerida.

\end{itemize}


\subsection{Lineas de trabajo a futuro}

\begin{itemize}
\item Se deberá desarrollar la aplicación móvil tanto para los sistemas operativos Android como iOS con el fin de llegar a la mayor cantidad de usuarios posibles, para lo cual se empleará el lenguaje Flutter debido a la portabilidad que el mismo ofrece.

\item Además, se separará la capa de operaciones del administrador de la del usuario en 2 microservicios diferentes, para discriminar la responsabilidad y potenciar la escalibilidad.

\item En lo referente a la escalabilidad, adicionalmente, se utilizará el orquestador de contenedores llamado Kubernetes, para mejorar el flujo de despliegue ante la posible alta concurrencia.

\item Se desarrollará un frontend para el backend ya creado con el fin de que los usuarios administradores puedan gestionar contenidos con mayor facilidad y mejor experiencia de usuario.

\item Se invertirán esfuerzos en continuar entrenando el chatbot con diferentes fuentes de datos para que éste brinde respuestas con mayor, más fehaciente y precisa información referente a la EM.

\item  Para optimizar el chatbot, además, se utilizará el alojamiento Pinecode para almacenar los vectores, logrando así una búsqueda más eficiente y célere.
\end{itemize}