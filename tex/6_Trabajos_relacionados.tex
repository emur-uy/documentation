\capitulo{6}{Trabajos relacionados}

Con la constante expansión y alcance en crecimiento que tiene la tecnología, cada vez son más los proyectos de software que nacen con la finalidad de resolver los problemas del hombre moderno y satisfacer sus necesidades en los distintos ámbitos de la vida: finanzas, educación, entretenimiento, desarrollo personal, salud. A continuación, se presentan y comparan algunos trabajos con objetivos similares a los planteados para el presente proyecto.

\subsection{Comparativa con otros proyectos}
En esta sección se muestran trabajos realizados hasta la fecha que guardan similitud con el presente proyecto por los objetivos a los que aspiran.

\subsubsection{“Desarrollo de una aplicación móvil para el seguimiento y monitorización en personas con esclerosis múltiple”.}
Se trata de un trabajo de fin de grado para la carrera de ingeniería informática, enfocado en el desarrollo de una aplicación móvil dirigida a pacientes con esclerosis múltiple cuyo propósito fue ofrecer un medio de participación en el control de su enfermedad en conjunto con sus doctores y especialistas, incluyendo funciones como control de medicación, gestión de citas, gestión de documentos, entre otras. Las características de este proyecto se muestran en la tabla ~\ref{tab:comp_APP}, en la columna denominada “EM1”.
\href{https://riunet.upv.es/handle/10251/188415}{Enlace al trabajo}

\subsubsection{“Diseño y desarrollo de una aplicación web para el seguimiento y control de la sintomatología en pacientes con Esclerosis Múltiple (EM)”.}
En este caso se trató de un proyecto presentado en la Universidad Oberta de Catalunya, enfocado en la creación de una aplicación web para pacientes con EM que les permitiese agilizar los tiempos de consulta mediante la gestión de formularios digitales que recojan sus síntomas y demás datos de interés. Las características de este proyecto se muestran en la tabla ~\ref{tab:comp_APP}, en la columna denominada “EM2”.
\href{https://openaccess.uoc.edu/handle/10609/147240}{Enlace al trabajo}

\subsubsection{“Diseño, estudio e implementación de una aplicación móvil para pacientes con esclerosis múltiple”.}
Proyecto desarrollado en la Universidad de Lleida, enfocado en el diseño e implementación de una aplicación móvil que les permitiese a los pacientes con EM llevar un control y seguimiento de su enfermedad y tratamiento mediante el control de eventos, como citas médicas, y comunicación con especialistas. Las características de este proyecto se muestran en la tabla ~\ref{tab:comp_APP}, en la columna denominada “EM3”.
\href{https://repositori.udl.cat/handle/10459.1/69628}{Enlace al trabajo}

\newpage
\begin{table}[]
\centering
\resizebox{\textwidth}{!}{%
\begin{tabular}{|c|c|c|c|c|}
\hline
\rowcolor[HTML]{C0C0C0} 
\color[HTML]{333333} \textbf{Funcionalidad} & \textbf{EM1} & \textbf{EM2} & \textbf{EM3} & \textbf{RedEM} \\
\rowcolor[HTML]{FFFFFF} \textbf{Registro de usuario} & \textbf{\checkmark} & \textbf{\checkmark} & \textbf{\checkmark} & \textbf{\checkmark}\\
\rowcolor[HTML]{FFFFFF} \textbf{Chatbot} & \textbf{} & \textbf{} & \textbf{} & \textbf{\checkmark}\\
\rowcolor[HTML]{FFFFFF} \textbf{Consejos, noticias o artículos de interés} & \textbf{\checkmark} & \textbf{} & \textbf{} & \textbf{\checkmark}\\
\rowcolor[HTML]{FFFFFF} \textbf{Diario de control de síntomas y medicación} &  \textbf{\checkmark} & \textbf{\checkmark} & \textbf{\checkmark} & \textbf{\checkmark}\\
\rowcolor[HTML]{FFFFFF} \textbf{Calendario para citas médicas} &  \textbf{\checkmark} & \textbf{} & \textbf{\checkmark} & \textbf{\checkmark}\\
\rowcolor[HTML]{FFFFFF} \textbf{Gráficos de seguimiento} &  \textbf{} & \textbf{\checkmark} & \textbf{} & \textbf{}\\
\rowcolor[HTML]{FFFFFF} \textbf{Planes de ejercicio físico} &  \textbf{\checkmark} & \textbf{} & \textbf{\checkmark} & \textbf{\checkmark}\\
\rowcolor[HTML]{FFFFFF} \textbf{Sistema de alarmas y recordatorios} &  \textbf{\checkmark} & \textbf{} & \textbf{\checkmark} & \textbf{\checkmark}\\
\rowcolor[HTML]{FFFFFF} \textbf{Guardar y compartir informes médicos} &  \textbf{\checkmark} & \textbf{\checkmark} & \textbf{} & \textbf{\checkmark}\\
\rowcolor[HTML]{FFFFFF} \textbf{Recetas alimenticias} &  \textbf{} & \textbf{} & \textbf{} & \textbf{\checkmark}\\
\rowcolor[HTML]{FFFFFF} \textbf{Preguntas y respuestas} &  \textbf{} & \textbf{} & \textbf{} & \textbf{\checkmark}\\
\rowcolor[HTML]{FFFFFF} \textbf{Mapa de servicios} &  \textbf{} & \textbf{} & \textbf{} & \textbf{\checkmark}\\
\rowcolor[HTML]{FFFFFF} \textbf{Clima} &  \textbf{} & \textbf{} & \textbf{} & \textbf{\checkmark}\\
\rowcolor[HTML]{FFFFFF} \textbf{Comparación y análisis de resultados} &  \textbf{} & \textbf{} & \textbf{} & \textbf{\checkmark}\\
\hline
\end{tabular}}%
\caption{Comparativa de aplicaciones de Esclerosis Múltiple.}
\label{tab:comp_APP}
\end{table}

\subsubsection{Principales fortalezas de \emph{RedEM}}
\begin{itemize}
\tightlist
\item
Proporciona al usuario recetas alimenticias, sección de preguntas y respuestas, e información del clima para el cuidado del paciente. 
\item 
Ofrece un mapa de servicios que le permite al paciente acceder a servicios de terceros que pueden ayudarle en situaciones concretas.
\item 
Dispone de un chatbot con el cual los pacientes podrán interactuar accediendo a información fidedigna de la Asociación para poder evacuar sus dudas ágilmente.
\item
Proporciona información actualizada, consejos y noticias referentes a la EM.
\item 
\emph{RedEM} operará utilizando un servidor que le permite atender a múltiples requests de forma simultánea sin que afecte su rendimiento.
\end{itemize}
