% Caso de Uso 49 -> Crear un servicio en el mapa
\begin{table}[p]
	\centering
	\begin{tabularx}{\linewidth}{ p{0.2\columnwidth} p{0.8\columnwidth} }
		\toprule
		\textbf{CU-49}    & \textbf{Crear un servicio en el mapa}\\
		\toprule
		\textbf{Versión}              & 1.0    \\
		\textbf{Autor}                & Roberto Fabián Delgado Pense \\
		\textbf{Requisitos asociados} & RF-13.1 \\ 
		\textbf{Descripción}          & Permite a los usuario administradores crear servicios en el mapa. \\
		\textbf{Precondición}         & El usuario tiene permiso de administrador. \\  
		\textbf{Acciones}             &
		\begin{enumerate}
			\def\labelenumi{\arabic{enumi}.}
			\tightlist
			\item El sistema recibe una solicitud HTTP para crear un servicio en el mapa.
                \item El sistema verifica que el usuario que realiza la solicitud se encuentre autenticado.
                \item El sistema valida la solicitud y extrae el nombre, latitud, longitud, tipo, horas disponible y teléfono.
                \item El sistema crea el nuevo servicio del mapa en la base de datos.
                \item El sistema devuelve una respuesta HTTP con el código 200 (OK) y los datos insertados.         
            \end{enumerate}\\
		\textbf{Postcondición}        & El servicio del mapa fue creado en la base de datos. \\
		\textbf{Excepciones}          & 
              \begin{enumerate}
			\def\labelenumi{\arabic{enumi}.}
			\tightlist
   			\item Si el usuario no se encuentra autenticado, o no es usuario administrador, el sistema devuelve una                 respuesta HTTP con el código 401 (No autorizado) y un mensaje.
                \item   Si el servicio no pudo ser creado en la base de datos, el sistema devuelve un                           una respuesta HTTP con el código 500 (Error interno del servidor) y un mensaje. 
            \end{enumerate}\\
		\textbf{Importancia}          & Alta \\
		\bottomrule
	\end{tabularx}
	\caption{CU-49 Crear un servicio en el mapa.}
\end{table}

% Caso de Uso 50 -> Obtener todos los servicios para el mapa
\begin{table}[p]
	\centering
	\begin{tabularx}{\linewidth}{ p{0.2\columnwidth} p{0.8\columnwidth} }
		\toprule
		\textbf{CU-50}    & \textbf{ Obtener todos los servicios del mapa}\\
		\toprule
		\textbf{Versión}              & 1.0    \\
		\textbf{Autor}                & Roberto Fabián Delgado Pense \\
		\textbf{Requisitos asociados} & RF-13.2 \\ 
		\textbf{Descripción}          & Permite a los usuarios obtener todos los servicios del mapa. \\
		\textbf{Precondición}         & El usuario se encuentra autenticado. \\  
		\textbf{Acciones}             &
		\begin{enumerate}
			\def\labelenumi{\arabic{enumi}.}
			\tightlist
			\item El sistema recibe una solicitud HTTP para obtener todos servicios del mapa.
                \item El sistema verifica que el usuario que realiza la solicitud se encuentre autenticado.
			\item El sistema obtiene todos los servicios del mapa.
                \item El sistema devuelve una respuesta HTTP con un código 200 (OK) y los servicios del mapa.         
            \end{enumerate}\\
		\textbf{Postcondición}        & -  \\
		\textbf{Excepciones}          &  Si el usuario no se encuentra autenticado, el 
                    sistema devuelve una respuesta HTTP con el código 401 (No autorizado) y un mensaje.\\
		\textbf{Importancia}          & Media \\
		\bottomrule
	\end{tabularx}
	\caption{CU-50  Obtener todos los servicios del mapa.}
\end{table}

% Caso de Uso 51 -> Actualizar los servicios del mapa
\begin{table}[p]
	\centering
	\begin{tabularx}{\linewidth}{ p{0.21\columnwidth} p{0.71\columnwidth} }
		\toprule
		\textbf{CU-51}    & \textbf{Actualizar los servicios del mapa}\\
		\toprule
		\textbf{Versión}              & 1.0    \\
		\textbf{Autor}                & Roberto Fabián Delgado Pense \\
		\textbf{Requisitos asociados} & RF-13.3 \\ 
		\textbf{Descripción}          & Permite al usuario administrador actualizar un servicio del mapa. \\
		\textbf{Precondición}         & El usuario tiene permisos de administrador. \\
		\textbf{Acciones}             &
		\begin{enumerate}
			\def\labelenumi{\arabic{enumi}.}
			\tightlist
                \item El sistema verifica que el usuario que realiza la solicitud se encuentre autenticado.
			\item El usuario ha proporcionado un identificador único (UUID), de servicio existente.
                \item El sistema extrae el nombre, latitud, longitud, tipo, horas disponible y teléfono.
                \item El sistema actualiza los datos del servicio con los nuevos valores proporcionados en la base de datos.
                \item El sistema devuelve una respuesta HTTP con un código 200 (OK) y un mensaje.
            \end{enumerate}\\
		\textbf{Postcondición}        & El servicio del mapa se actualiza correctamente en la base de datos.\\
		\textbf{Excepciones}          & 
            \begin{enumerate}
			\def\labelenumi{\arabic{enumi}.}
			\tightlist
   			\item Si el usuario no se encuentra autenticado, o no es del tipo usuario administrador, el sistema devuelve una                 respuesta HTTP con un código 401 (No autorizado) y un mensaje.
                \item   Si el servicio no se encuentra en la base de datos, el sistema devuelve un                           una respuesta HTTP con un código 404 (No encontrado) y un mensaje. 
                \item Si ocurre un error al actualizar el servicio, el sistema devuelve una respuesta con código de estado 500 (Error interno del servidor) y un mensaje.
            \end{enumerate}\\
		\textbf{Importancia}          & Alta \\
		\bottomrule
	\end{tabularx}
	\caption{CU-51 Actualizar los servicios del mapa.}
\end{table}

% Caso de Uso 52 -> Eliminar un servicio del mapa
\begin{table}[p]
	\centering
	\begin{tabularx}{\linewidth}{ p{0.21\columnwidth} p{0.71\columnwidth} }
		\toprule
		\textbf{CU-52}    & \textbf{Eliminar un servicio del mapa}\\
		\toprule
		\textbf{Versión}              & 1.0    \\
		\textbf{Autor}                & Roberto Fabián Delgado Pense \\
		\textbf{Requisitos asociados} & RF-13.4 \\ 
		\textbf{Descripción}          & Permite al usuario administrador eliminar un servicio del mapa. \\
		\textbf{Precondición}         & El usuario ha proporcionado un identificador único (UUID) de servicio existente.\\
		\textbf{Acciones}             &
		\begin{enumerate}
			\def\labelenumi{\arabic{enumi}.}
			\tightlist
			\item El sistema recibe una solicitud HTTP para eliminar un servicio del mapa.
                   \item El sistema verifica que el usuario que realiza la solicitud se encuentre autenticado y tenga el permiso de usuario administrador.
			\item El sistema extrae el identificador único (UUID) del servicio desde la URL de la solicitud.
                \item El sistema busca en la base de datos el servicio usando el indentificador único (UUID).
                \item El sistema elimina el servicio de la base de datos.
                \item El sistema devuelve una respuesta HTTP con un código 200 (OK) y un mensaje.
            \end{enumerate}\\
		\textbf{Postcondición}        & El servicio fue eliminado de la base de datos.\\
		\textbf{Excepciones}          & 
            \begin{enumerate}
			\def\labelenumi{\arabic{enumi}.}
			\tightlist
   			\item Si el usuario no se encuentra autenticado, o no es usuario administrador, el sistema devuelve una                 respuesta HTTP con un código 401 (No autorizado) y un mensaje.
                \item   Si el servicio no se encuentra en la base de datos, el sistema devuelve un                           una respuesta HTTP con un código 404 (No encontrado) y un mensaje. 
                \item Si ocurre un error al eliminar el servicio, el sistema devuelve una respuesta con código de estado 500 (Error interno del servidor) y un mensaje
            \end{enumerate}\\
		\textbf{Importancia}          & Alta \\
		\bottomrule
	\end{tabularx}
	\caption{CU-52 Eliminar un servicio del mapa.}
\end{table}

% Caso de Uso 53 -> Chatbot
\begin{table}[p]
	\centering
	\begin{tabularx}{\linewidth}{ p{0.21\columnwidth} p{0.71\columnwidth} }
		\toprule
		\textbf{CU-53}    & \textbf{Chatbot}\\
		\toprule
		\textbf{Versión}              & 1.0    \\
		\textbf{Autor}                & Roberto Fabián Delgado Pense \\
		\textbf{Requisitos asociados} & RF-14.1 \\ 
		\textbf{Descripción}          & Permite a cualquier usuario interactuar con el chatbot. \\
		\textbf{Precondición}         & -\\
		\textbf{Acciones}             &
		\begin{enumerate}
			\def\labelenumi{\arabic{enumi}.}
			\tightlist
			\item El sistema recibe una solicitud HTTP con una pregunta del usuario.
                \item El sistema devuelve una respuesta HTTP con un código 200 (OK) y la respuesta.
            \end{enumerate}\\
		\textbf{Postcondición}        & -\\
		\textbf{Excepciones}          & 
            \begin{enumerate}
			\def\labelenumi{\arabic{enumi}.}
			\tightlist
                \item Si ocurre un error al eliminar el servicio, el sistema devuelve una respuesta con código de estado 500 (Error interno del servidor) y un mensaje.
            \end{enumerate}\\
		\textbf{Importancia}          & Alta \\
		\bottomrule
	\end{tabularx}
	\caption{CU-53 Chatbot}
\end{table}