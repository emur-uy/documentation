\capitulo{1}{Introducción}

\subsection{Encuadre del TFG}
La presente entrega se enmarca en la formación de Grado en Ingeniería Informática impartida por la Universidad de Burgos de España, constituyendo la última instancia evaluativa requerida para la aprobación de la misma.
En tanto trabajo final de grado (TFG), su propósito radica en propiciar la articulación de los contenidos abordados conforme avanzó el ciclo educativo y valorar la incorporación de las competencias adquiridas para el egreso.

Amerita mencionar que la redacción presentada a continuación está realizada en castellano rioplatense, un derivación de la lengua española hablada en algunos países de latinoamérica, puesto que mi nacionalidad y lugar de residencia es Uruguay.

\subsection{Introducción al TFG}
Una aplicación móvil, también conocida como app, por el vocablo application en idioma inglés, es un software diseñado para ser utilizado en dispositivos móviles como smartphones o tablets, bien sean de sistemas operativos iOS o Android. Estas aplicaciones pueden ser descargadas e instaladas desde tiendas virtuales como Google Play o Apple Store y ofrecen una amplia variedad de funciones y servicios. Asimismo, pueden ser clasificadas en diferentes categorías; a saber: juegos, redes sociales, productividad, entre otras. De esta forma, las aplicaciones móviles pueden ser utilizadas en una amplia variedad de campos tales como la educación, la salud, el comercio electrónico y el entretenimiento.

Dentro del ámbito de la salud, las aplicaciones móviles han revolucionado la forma en que los pacientes interactúan con los proveedores de atención médica y gestionan su propia salud. Estas aplicaciones pueden ser utilizadas para diversos propósitos, desde el seguimiento de la administración de medicamentos hasta la monitorización de la actividad física y la gestión de enfermedades crónicas. Ello permite, a pacientes y profesionales, tener acceso a información y servicios en línea, siendo más eficientes, por lo que representan una herramienta muy valiosa y especialmente útil para personas con enfermedades de curso crónico como lo es la Esclerosis Múltiple (en adelante EM), que a menudo necesitan navegar por un sistema de atención sanitaria complejo y fragmentado a lo largo de las diferentes etapas de su ciclo vital.

La EM es una enfermedad neurológica que afecta al Sistema Nervioso Central, de origen multicausal, pero cuyo factor desencadenante aún no ha sido descubierto. Se trata de una enfermedad inflamatoria, autoinmune, y potencialmente discapacitante.

Se ha identificado que la progresión deviene con el paso del tiempo en todos los tipos de EM desde las primeras etapas de la enfermedad, incluso si el individuo no experimenta síntomas, siendo algunos de los más frecuentes:

\begin{itemize}
\tightlist
\item
    visión doble y/o borrosa 
\item
    dificultades en el habla
\item
    parestesias 
\item
    espasticidad 
\item
    afectación cognitiva     
\item   
    entumecimiento 
\item   
    ataxia 
\item   
    fatiga
\item   
    dolores musculares
\item   
    pérdida de fuerza
\item   
    disminución de la sensibilidad
\item   
    afectación a nivel urológico
\end{itemize}

El diagnóstico de una patología crónica y de constante progresión, así como la amplia diversidad de síntomas y la incertidumbre de la evolución, son aspectos que suelen tener un impacto multidimensional (biológico, cognitivo, psicoemocional, socio-ecológico, espiritual) significativo en la calidad de vida de las personas que conviven con EM, quienes suelen enfrentar muchos desafíos en el proceso de diagnóstico, tratamiento y seguimiento de su enfermedad. 

Desde el momento en que son diagnosticados, los pacientes pueden necesitar coordinar citas con múltiples especialistas, obtener recetas y medicamentos, y conocer en profundidad los servicios ofrecidos por su prestador de salud. Asimismo, a medida que la enfermedad progresa, pueden necesitar acceder a diferentes tipos de terapias y tratamientos. 

Esta enfermedad, que afecta a más de 2.8 millones de personas en todo el mundo, presenta, en Uruguay, el índice de prevalencia más alto de Latinoamérica  y, reporta 94 casos nuevos por año en el país, de acuerdo al estudio multicéntrico EMELAC culminado en 2022.

En este contexto, el desarrollo de una aplicación móvil específica para ayudar a las personas con EM a moverse por el sistema sanitario puede ser una herramienta muy efectiva. Esta aplicación podría ser utilizada para integrar diferentes servicios de salud, como citas médicas, pruebas diagnósticas, tratamientos y seguimiento sintomático, en una sola plataforma. De esta manera, los pacientes podrían acceder a los servicios de salud necesarios de manera más rápida y eficiente, lo que mejoraría su calidad de vida.

En relación con lo anterior, el backend de la aplicación que se propone en el presente proyecto permitirá a las personas con EM acceder, de manera más eficiente, a información respecto a su enfermedad y a los recursos necesarios para navegar por el sistema sanitario, asi como también los dispositivos existentes en la matriz de protección social.

Por lo tanto, y en definitiva, el desarrollo de una aplicación móvil focalizada en la experiencia del enfermar con EM, configura una oportunidad única para optimizar la calidad de vida de las personas afectadas por esta patología, asi como también de toda la sociedad uruguaya.

Esta aplicación tiene potencial para ser replicable y escalable a otras patologías así como también a otros contextos socio-económicos, culturales y geográficos, constituyendo una herramienta muy valiosa para integrar diferentes funcionalidades relativas a la salud en una sola plataforma. Asimismo, sienta el terreno fértil para la recolección de datos e información relevante que permita a tomadores de decisión mapear y perfilar la comunidad de personas con EM en el país para desarrollar soluciones acorde a sus necesidades.


